\begin{answer}

From problem statement, perfect calibration refers to the condition where the predicted probability of an event occurring matches the observed frequency of the event occurring. \\

Perfect accuracy means the model makes no errors in prediction. That is, every prediction matches the true label. \\

So perfect calibration doesn't imply prefect accuracy. For example, in a case $p = 0.5$, even when the model is perfectly calibrated, prediction could be wrong half of the time if the probabilities are equally distributed around this threshold. \\

Conversely, under problem setting of binary classification, if a model achieves perfect accuracy, it implies perfect calibration. Proof by contradiction, an inaccurate calibration will inevitably lead to some classification error.



\end{answer}
