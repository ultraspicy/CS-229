\begin{answer}

If $F(x_i) = y_i$, then $f(x_i)$ has the same sign of $y_i$, so $exp(-f(x_i) y_i)$ is in the range (0, 1), meaning when $F(x_i) = y_i$, 
\begin{equation}
		1_{\{F(x_i) \neq y_i\}} = 0 <  exp(-f(x_i) y_i)
\end{equation}

If $F(x_i) \ne y_i$, then $f(x_i)$ has the opposite sign of $y_i$, so $exp(-f(x_i) y_i)$ is great than 1, meaning when $F(x_i) = y_i$, 
\begin{equation}
		1_{\{F(x_i) \neq y_i\}} = 1 <  exp(-f(x_i) y_i)
\end{equation}

Sum over $n$ observations, we got 
\begin{equation}
    \varepsilon_{\text{training}}
		:= \frac{1}{n} \sum_{i=1}^n 1_{\{F(x_i) \neq y_i\}}
		\le \frac{1}{n} \sum_{i=1}^n \exp(-f(x_i) y_i),
\end{equation}
\end{answer}