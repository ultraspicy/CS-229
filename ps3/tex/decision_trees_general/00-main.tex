\item \points{35} {\bf Decision trees}

Consider the problem of predicting if a person has a college degree based on age and salary. Table~\ref{tab:decisionTree} contains training data for 10 individuals. 
\begin{table}[h!]
	\centering
	\begin{tabular}{c|c|c}
		\hline
		Age & Salary (\$1k) &  College degree\\\hline
		24& 40 &  Yes\\
		53& 52 &  No\\
		23&  25&  No\\
		25&  77&  Yes\\
		32&  48&  Yes\\
		52&  110&  Yes\\
		22&  38&  Yes\\
		43&  44&  No\\
		52&  27&  No\\
		48&  65& Yes\\ \hline
	\end{tabular}
	\caption{Training data for predicting college degree.}
	\label{tab:decisionTree}
\end{table}

For questions below, the answers may not be unique. Any plausible solution is acceptable. Keep two significant decimals in part (a) and (c).


\begin{enumerate}
    \item \subquestionpoints{5} Build a decision tree for classifying whether a person has a college degree by greedily choosing threshold splits that minimize the classification error. Provide a list of all splits and the classification error reduction at each split.
    
	\ifnum\solutions=1 {
	\begin{answer}
\begin{figure}[H]
    \centering
    \includegraphics[width=0.5\linewidth]{Screenshot 2024-02-19 at 20.26.53.png}
    \caption{ps3::q3::(a)}
    \label{fig:enter-label}
\end{figure}
\end{answer}
        } \fi
        
    \item \subquestionpoints{15} Now let's implement a classification, univariate decision tree with misclassification loss (mentioned in equation \ref{misclassificationloss}). The starter code is provided in \url{src/decision_trees_general/decision_tree.py}. Fill in the functions marked with \textbf{\#TODO}. You are not allowed to use any package other than NumPy. You \textbf{cannot} assume there are only two classes. \textbf{Deliverables}: report the accuracy output when running the Python script. For reference, the staff solution gives the same expected accuracy in part (a) for the college degree dataset (Table \ref{tab:decisionTree}) and 93.33\% for the iris dataset.
    
	\ifnum\solutions=1 {
	\begin{answer}

Accuracy for college degree dataset: 90.0\% \\
Accuracy for iris dataset: 93.33\% \\
\end{answer}
        } \fi
    
    \item \subquestionpoints{5} List out the cases where Gini loss will stay the same after a split.  Show why these do not violate the strong concavity of the Gini loss.  Briefly explain why these cases do not prevent a fully grown tree from achieving zero Gini loss. (\textbf{Hint}: Recall the definition of strict concavity).
    
	\ifnum\solutions=1 {
	\begin{answer}

Given
\begin{equation}
    \alpha_{i, (t+1)}
 			= \frac{\alpha_{i, t} \exp(-\hat{w}_t f_t(x_i) y_i)}{Z_t},
\end{equation}
From induction,
\begin{equation}
    \alpha_{i, (t+1)}
 			= \frac{\exp(-\hat{w}_t f_t(x_i) y_i)}{Z_t} \cdot \frac{\exp(-\hat{w}_{t-1} f_{t-1}(x_i) y_i)}{Z_{t-1}} \cdots \alpha_{i, 1}
\end{equation}

Given $\alpha_{i, 1}$ is $\frac{1}{n}$, the above equation can be written as 

\begin{equation}
    \alpha_{i, (t+1)} = \frac{1}{n} \prod_{t=1}^T \frac{\exp(-\hat{w}_t f_t(x_i) y_i)}{Z_t}
\end{equation}

equivalently, 
\begin{equation}
    \prod_{t=1}^T Z_t \cdot \alpha_{i, (t+1)} \cdot n= \prod_{t=1}^T \exp(-\hat{w}_t f_t(x_i) y_i)
\end{equation}

Given the definition of $f(x)$, we have the following 
\begin{equation}
    \exp(-f(x_i)y_i) = \exp\left(-\sum_{t=1}^{T} \hat{w}_t f_t(x_i)y_i\right) = \prod_{t=1}^{T} \exp\left(-\hat{w}_t f_t(x_i)y_i\right)
\end{equation}


Combine (19) and (20), 
\begin{equation}
    \exp(-f(x_i)y_i) = \prod_{t=1}^T Z_t \cdot \alpha_{i, (t+1)} \cdot n
\end{equation}

Sum all $n$ observations, 
\begin{equation}
    \sum_{i=1}^{n} \exp(-f(x_i)y_i) = \sum_{i=1}^{n} \alpha_{i, (t+1)} \cdot \prod_{t=1}^T Z_t \cdot n
\end{equation}

$\alpha_{i, (t+1)}$ is a normalized parameter, so 
\begin{equation}
    \sum_{i=1}^{n} \alpha_{i, (t+1)} = 1
\end{equation}

Replacing (23) into (22), we get
\begin{equation}
    \frac{1}{n} \sum_{i=1}^n \exp(-f(x_i)y_i)= \prod_{t=1}^T Z_t
\end{equation}
\end{answer}
        } \fi
        
    \item \subquestionpoints{4} Multivariate decision trees have practical advantages and disadvantages. List two advantages and two disadvantages multivariate decision trees have compared to univariate decision trees.
    
	\ifnum\solutions=1 {
	\begin{answer}

Advantage of multivariate 1) it can deal with more than one variable at the same time, meaning a more complex hyper plane can be denoted by split, thus more powerful the model and be and thus less bias of the model. 2) it can reduce the over fitting in a sense that it needs to consider more variable in one split, so it is less prone to one single dimension of data

Disadvantage 1) it is more computationally expensive in both training and fit process 2) it is more difficult to let user to interpret the result 
\end{answer}
        } \fi
        
    \item \subquestionpoints{4} 

Bagging, short for "bootstrap aggregating," is a powerful ensemble learning technique that aims to improve the stability and accuracy of machine learning algorithms. It leverages the concept of bootstrapping, which involves simulating the drawing of a new sample from the true underlying distribution of the training set, as the training set is presumed to be a representative sample of the true distribution. In practice, this is done by generating new datasets through uniform sampling with replacement from the original dataset.

The "aggregating" component of bagging comes into play by repeating this bootstrapping process for each model in the ensemble, allowing each to be trained independently on a unique dataset. When considering decision trees, the method's utility becomes evident as it mitigates overfitting by ensuring that each tree in the ensemble is exposed to different subsets of the training data. This reduces the likelihood that the ensemble will fixate on particular data points, thus lowering overall variance. Statistically, each bootstrapped sample will contain, in expectation, about $1 - \frac{1}{e} \approx 63.2\%$ of unique data points from the original dataset.

However, the effectiveness of bagging depends on the characteristics of the underlying models. For models with low variance (and typically high bias), bagging may produce very similar models, which diminishes its benefits. On the other hand, with high-variance models such as decision trees, bagging capitalizes on the models' instability to promote diversity in the ensemble, thereby enhancing its performance. This results in an ensemble that maintains low bias while reducing variance, leading to a robust aggregate model.

Consider a training set X. In bootstrap sampling, each time we draw a random sample $Z$ of size N from the training data and obtain ${Z_1, Z_2, ..., Z_B}$ after $B$ times, i.e. we generate B different bootstrapped training data sets. If we apply bagging to regression trees, each time a tree $T_i (i = 1,2,...,B)$ is grown based on the bootstrapped data $Z_i$, and we average all the predictions to get:
\begin{align*}
    \hat{T(x)} =  \frac{1}{B}\sum_{i=1}^{B} T_i(x)
\end{align*}
Now, if $T_1, T_2,..., T_B$ is independent from each other, but each has the same variance $\sigma^2$, the variance of the average $\hat{T}$ is $\sigma^2/B$. However, in practice, the bagged trees could be similar to each other, resulting in correlated predictions. Assume $T_1, T_2,..., T_B$ still share the same variance $\sigma^2$, but have a positive pair-wise correlation $\rho$. We define the correlation between two random variables as:\\
\begin{align*}
    Corr(X,Y)=\dfrac{Cov(X,Y)}{\sqrt{Var(X)}\sqrt{Var(Y)}}
\end{align*}
Thus, we have $\rho = Corr(T_i(x), T_j(x)), i \neq j$.

Show that in this case, the variance of the average is given by:
\begin{align*}
    Var(\frac{1}{B}\sum_{i=1}^{B} T_i(x)) = \rho \sigma^2 + \frac{1-\rho}{B} \sigma^2
\end{align*}
    
	\ifnum\solutions=1 {
	\begin{answer}

(i) 
For regression and predict the salary from college degree and age, we can output the average salary that falls into this node. \\

Changes to make - 1)change the output from a predicted\_class to the average of salary of all its data 2) All reference of node.predicted\_class needs to be corrected to be node.aver\_salary \\
pseudo code as follows\\
    \hspace{10mm} (commented out) predicted\_class = np.argmax(num\_samples\_per\_class) \\
    \hspace{10mm} (commented out) root\_node = Node(predicted\_class=predicted\_class) \\
    root\_node = Node(np.mean(data[index\_of\_salary]))

(ii) 
We still derive a threshold for each split, and minimize the sum of absolute difference between each data and the average of all data in the node. For example, data 1 3 2 fall in the same node, the cost is abs(1 - 2) +  abs(3 - 2) + abs(2 - 2) \\
Changes to make - modify \_misclassification\_loss method \\ 
pseudo code \\
    cost = 0 \\
    aver = mean(data) \\
    for d in data:  cost += abs(data - aver) \\
    return cost
\end{answer}
        } \fi
        
        
\end{enumerate}
