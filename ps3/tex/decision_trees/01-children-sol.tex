\begin{answer}

Given 
\begin{align*}
    G(R_m) = G(\vec{p}_m) = \sum_{k=1}^K p_{mk} (1 - p_{mk})
\end{align*}

We have 
\begin{equation}
    \frac{\partial}{\partial p_{mi}} G(R_m)= 1 - 2p_{mi}
\end{equation}

\begin{equation}
    \frac{\partial ^2}{\partial p_{mi}^2} G(R_m)= -2
\end{equation}

So $G(R_m)$ is strictly concave. For binary classification, a parent node with a proportion $p$ of class 1, thus the Gini loss is $G(p)$. After a split, we have two child nodes with proportions $p_1$ and $p_2$ for class 1, and size of region $\lvert R_1 \rvert$ and $\lvert R_2 \rvert$ respectively. 

Let 
\begin{equation}
    t = \frac{\lvert R_1 \rvert}{\lvert R_1 \rvert + \lvert R_2 \rvert}, \text{then 
 } 1- t = \frac{\lvert R_2 \rvert}{\lvert R_1 \rvert + \lvert R_2 \rvert}
\end{equation}

then 
\begin{equation}
t G(p_1) + (1-t) G(p_2) \le G(tp_1 + (1-t)p_2) = G(p)
\end{equation}

The first inequity in (5) is due to concave function $G()$, the second equation is due to $p$ is the weighted average of $p_1$ and $p_2$. And (5) just shows that the weighted Gini loss of the children can not exceed that of the parent
\end{answer}