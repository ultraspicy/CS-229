\item \points{12} {\bf Double Descent on Linear Models}

In this question, you will empirically observe the sample-wise double descent phenomenon. While this problem is self-contained, additional information on this phenomenon (as well as the related model-wise double descent phenomenon) is available in section 8.2 of the lecture notes. 

In the sample-wise double descent phenomenon, the test losses of some learning algorithms or estimators do not monotonically decrease as we have more training examples, but instead have a curve with two U-shaped parts. The double descent phenomenon can be observed even for simple linear models. In this question, we consider the following setup. Let $\{(x^{(i)},y^{(i)})\}_{i=1}^{n}$ be the training dataset. Let $X\in\R^{n\times d}$ be the matrix representing the inputs (i.e., the $i$-th row of $X$ corresponds to $x^{(i)})$), and $\vec{y}\in\R^{n}$ the vector representing the labels (i.e., the $i$-th row of $\vec{y}$ corresponds to $y^{(i)})$):
$$
X=
\begin{bmatrix}
	- & x^{(1)} & - \\
	- & x^{(2)} & - \\
	\vdots & \vdots & \vdots\\
	- & x^{(n)} & - 
\end{bmatrix},\qquad
\vec{y}=
\begin{bmatrix}
	y^{(1)} \\
	y^{(2)}\\
	\vdots\\
	y^{(n)}
\end{bmatrix}.
$$
Similarly, we use $X_v\in \R^{m\times d}, \vec{y}_v\in \R^{m}$ to represent the test dataset, where $m$ is the size of the test dataset. We assume that the data are generated with $d=500$. 

In this question, we consider \emph{regularized} linear regression. For a regularization level $\lambda\ge 0$, define the regularized cost function $$J_\lambda(\beta)=\frac{1}{2}\|X\beta-\vec{y}\|_2^2+\frac{\lambda}{2}\|\beta\|_2^2,$$ and its minimizer $\hat{\beta}_\lambda=\arg\min_{\beta\in \R^{d}}J_\lambda(\beta).$

\begin{enumerate}
 	\input{doubledescent/01-solution}
	\ifnum\solutions=1 {
	\begin{answer}

Given 
\begin{equation}
    J_\lambda(\beta)=\frac{1}{2}\|X\beta-\vec{y}\|_2^2+\frac{\lambda}{2}\|\beta\|_2^2    
\end{equation}

then
\begin{equation}
    \nabla J_\lambda(\beta) =  X^\top X \beta -  X^\top y + \lambda \beta
\end{equation}

let $\nabla J_\lambda(\beta) = 0$, we get 

\begin{equation}
     X^\top X \beta + \lambda \beta = X^\top y = (X^\top X + \lambda I_{d \times d}) \beta
\end{equation}

then 
\begin{equation}
    \hat{\beta}_\lambda=(X^\top X+\lambda I_{d\times d})^{-1}X^\top \vec{y}
\end{equation}
\end{answer}

} \fi

\input{doubledescent/02-unreg}
\ifnum\solutions=1 {
	
\begin{answer}

\begin{figure}[H]
    \centering
    \includegraphics[width=0.5\linewidth]{unreg.png}
    \caption{ps1\_q5\_(b)}
    \label{fig:enter-label}
\end{figure}
\end{answer}

} \fi

	\input{doubledescent/03-reg}
	\ifnum\solutions=1 {
	
\begin{answer}
\begin{figure}
    \centering
    \includegraphics[width=0.5\linewidth]{reg.png}
    \caption{ps1\_q5_(c)}
    \label{fig:enter-label}
\end{figure}
\end{answer}

	} \fi

\end{enumerate}
